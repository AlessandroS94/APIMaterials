In this work we presented a different approach to face the problem of the API recommendations. This area, as we saw from the relevant number of related work, is growing in importance and in the literature there are a lot of approaches to solve it. Our work is based on existing tools but combined in an organic architecture built in Eclipse. On the one side, we have Simian, a code cloner that is able to retrieve cloned code among files without preprocessing or postprocessing phases. On the other hand, we have CLAMS, a tool mentioned also in the related works that gives as output a ranked list of patterns. We can reuse these two approaches because they are completely free and available on Github, so we can takes the main ideas and combining them to realize something that bring novelty in this field. As told before, the final goal is to provide API function call recommendation in the context of software complex system; it means that we have to analyze Java user's file put within a more bigger structure (as Eclipse project for example). So, a code cloner tool is not enough to grant a proper recommendation at the level of snippet of code and we adding the concept of patterns thanks to CLAMS approach. From this, we reuse the output files that contains useful patterns for our purposes. In this way, we provide recommendations at level of code snippet, that is closer to the code and helps the developer to complete or to have a possible different implementation of the feature that he is developing. Notice that the provided recommendations are always related to a certain library at once. \\
Of course, our approach leaves some issues opened, as the precision of the mined snippet of code (like false positive in the Simian analysis or in the CLAMS patterns). From the evaluation framework, in fact, we can see that the approach is quite precise in average but it is strongly related to the developer's snippet code that Simian takes in input. From the comparison with PAM, we notice that the proposed approach is quite accurate and only in the case of some values of the recall and f measure it is worst rather than PAM. Another issue is that CLAMS patterns have a structure composed first by a list of variable used by the method calls and then the chain of method declaration to realize a specific feature. This structure can bring some bias because Simian could find similarities only among the list of variables and not in the method calls, those are strongly related with the API recommendations.\\ 
Moreover, Simian divides the source code to analyze into blocks, limited by the threshold parameter; in some cases this can lead false positive because Simian cannot identify a certain pattern in the code if it is not smaller than the threshold value. For example, in the source code there are two methods that belong to a CLAMS pattern which includes another method invocations; if in the original source code the third invocation differs from that, Simian is not able to retrieve the pattern. 
Some threats could be arise also from the evaluation framework in which we used Rascal. In order to validate the approach, we had set up an Eclipse structure as showed in the related section. Although the structure is correct, it is manually built so it can bring some problems that could affect the metrics, in particular the precision and recall. \\
From the point of view of improvements, we choose to use a code cloner but there are a lot of alternative approach that we can implements, like choose another code cloner or change completely the approach. We could beatify the output results, that are now presented in a simple file, maybe by setting up an Eclipse plugin, with all necessary components. Regarding the evaluation, we can provide a human survey by involving developers with different skills and knowledge to asses the provided results with a different perspective. 
Considering other approaches,many authors use probabilistic techniques, like define own models or probability distribution or, reversely, they rely completely on the AST techniques. The key point of our work is the use of a code cloner to detect similarities between the pattern retrieved by CLAMS and the actual code snippet. Although Simian doesn't have any pre or post processing phases inside, we avoid this limitations by select manually the source files and rank the results considering the duplicated lines of code. Moreover, we added also the AST concept in the evaluation framework of the results; in this way, we can add this feature to Simian, that consider only the textual similarity. Following this road, a possible improvement should be use a code cloner based on different similarity comparison (like suffix tree), but the challenge of integrate it in the Eclipse platform still remain opened. Most of code cloner are developed for simply analyze the developer's code and are used only to find clones in different files.   